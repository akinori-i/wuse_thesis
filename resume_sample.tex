%
% 卒論レジュメフォーマット Ver.2.0 pLaTeX版
%
\documentclass[twocolumn]{jarticle} % 2段組のスタイルを用いている

\usepackage{wuse_resume}
\usepackage{url}	% \url{}コマンド用.URLを表示する際に便利
%\usepackage[dvipdfmx]{graphicx}  % ←graphicx.styを用いてEPSを取り込む場合有効にする
			% 他のパッケージ・スタイルを使う場合には適宜追加

%%%%%%%%%%%%%%%%%%%%%%%%%%%%%%%%%%%%%%%%%%%%%%%%%%%%%%%%%%%%%%%%%%%%%%%%

%%
%% タイトル,学生番号,氏名などを設定する
%%

\タイトル{p\LaTeX によるレジュメの作成}
\研究室{ソーシャルソフトウェア工学}
\学生番号{60256000}
\氏名{伊原 彰紀}

\概要{%
この文章は,和歌山大学システム工学部システム工学科社会情報学メジャーの卒業研究発表会に
おいて配布されるレジュメをp\LaTeX を用いて作成するための方法を記述したものである.
この文章もまた,統一されたレジュメの体裁に沿って作成されており,レジュメ作成の際の参考にされたい.
}

\キーワード{レジュメ}
\キーワード{和歌山大学システム工学部}
\キーワード{卒業論文}

%%%%%%%%%%%%%%%%%%%%%%%%%%%%%%%%%%%%%%%%%%%%%%%%%%%%%%%%%%%%%%%%%%%%%%%%

%% 以下の3行は変更しない

\begin{document}
\maketitle
\thispagestyle{empty} % タイトルを出力したページにもページ番号を付けない

%%%%%%%%%%%%%%%%%%%%%%%%%%%%%%%%%%%%%%%%%%%%%%%%%%%%%%%%%%%%%%%%%%%%%%%%

%%
%% 本文 - ここから
%%

\section{レジュメ}

レジュメは,A4用紙2枚を上限とし,卒業論文提出時に同時に提出することに
なっている\footnote{平成15年度より,レジュメのスタイルは学科である程度統一されることになった.}.
レジュメに記述しなければならないのは,本文に加えて次の各項目である.

\begin{itemize}
  \item タイトル
  \item 所属研究室
  \item 学生番号
  \item 氏名
  \item 概要
  \item キーワード(5語程度)
  \item 参考文献
\end{itemize}

本フォーマットでは,2ページという限られた領域を有効に利用するために,2段組のレイアウトを採用して
いる.

本レジュメフォーマットは,
\begin{itemize}
  \item {\tt wuse\_resume.sty}
  \item {\tt resume\_sample.tex}
\end{itemize}
の2つのファイルから構成される.
{\tt wuse\_resume.sty}ファイルは,\TeX を用いてレジュメを作成する際に,統一的なスタイルを
与える
\LaTeX のスタイルファイルである.
{\tt resume\_sample.tex}ファイルは,このスタイルファイルを利用してレイアウトした文書の例であ
る.

以下では,本スタイルファイルを利用するための方法,および2段組のレイアウトを効果的に使うための手
法を解説する.
なお,本スタイルファイルの姉妹版として,
卒業論文用スタイルファイル\cite{wusethesis}
も用意しているので参考にされたい.
\TeX や\LaTeX についての情報は,書籍\cite{latex_j,latexcomp,latex2e}や
\TeX Wiki\footnote{\url{https://texwiki.texjp.org/}}が詳しい.

\section{タイトル,氏名,概要など}

タイトルや氏名などは,それぞれ``\verb|\タイトル|''や``\verb|\氏名|''コマンドを利用して指
定する.
この他に,
  所属研究室(\verb|\研究室|),
  学生番号(\verb|\学生番号|),
  概要(\verb|\概要|),
  キーワード(\verb|\キーワード|)
を指定するためのコマンドが用意されている.
これらの情報は,\verb|\maketitle|によって1ページ目上部中央に出力される.

\section{図,表}\label{chap:fig-tab-exp}

論文/レジュメでは,図,表などを効果的に使用する.

\subsection{図}

{\tt figure}環境を利用することによって図にキャプション(\verb|\caption|)を付けることができ
る.
図には通し番号が付けられ,キャプションに\verb|\label|を設定しておくと,
``図\ref{fig:sample}''のように\verb|\ref|によって図を番号で参照することができる.
図\ref{fig:sample}に{\tt figure}環境を用いた記述例を示す.
なお,図のキャプションは,図の下部に付けるのが一般的である.

\begin{figure}[t]
  \centering
  %% EPS形式の図を入れる場合には,以下のminipageを \includegraphics{hoge.eps}
  %% などに置き換える.
  \begin{minipage}{0.45\textwidth}
    \begin{verbatim}
	  \begin{figure}
	    \centering
	    図(\includegraphics{hoge.eps}など)
	    \caption{figure環境}\label{figenv}
	  \end{figure}
    \end{verbatim}
  \end{minipage}
  %% ここまで(置き換える範囲)
  \caption{図の例}\label{fig:sample}
\end{figure}

\verb|includegraphics|を用いてPDF形式/PNG形式/JPEG形式/EPS形式等の図を文章の中に取り込むことができる.
この場合には,\verb|\begin{document}|の前に\verb|\usepackage[dvipdfmx]{graphicx}|を追加する.

\subsection{表}

{\tt table}環境を利用することによって図と同じように,キャプションをつけたり,ラベルにより参照したり
することができる.
なお,表のキャプションは,表の上部に付けるのが一般的である.

\subsection{横長の図・表}

2段組の文書を作成していて,1段でレイアウトしたい横長の図や表を入れるためには,
\verb|figure*|や\verb|table*|環境を利用する.
表\ref{tab:long_sample}に横長の表の例を示す.

\begin{table*}
  \caption{横長の表}\label{tab:long_sample}
  \centering
  \begin{tabular}{lrrrr}
  \hline
  \multicolumn{1}{c}{条件} & \multicolumn{1}{c}{その1} & \multicolumn{1}{c}{その2} & \multicolumn{1}{c}{その3} & \multicolumn{1}{c}{その4}\\ 
  \hline
  条件1 & 3.14159265358979 & 3.14159265358979 & 3.14159265358979 & 3.14159265358979\\
  条件2 & 3.14159265358979 & 3.14159265358979 & 3.14159265358979 & 3.14159265358979\\
  条件3 & 3.14159265358979 & 3.14159265358979 & 3.14159265358979 & 3.14159265358979\\
  \hline
  \end{tabular}
\end{table*}

\section{参考文献}

レジュメには,参考文献も含める.
ここでは,一般的な\verb|thebibliography|環境を利用している.

\section{質問等}

このレジュメ体裁(p\LaTeX 版)に関する質問は,
メールにて,\verb|fukuyasu@wakayama-u.ac.jp|まで.

\section{おためし}

\subsection{箇条書}

\begin{itemize}
  \item 普通の箇条書1
  \item 普通の箇条書2
  \item 普通の箇条書3
  \item 長い箇条書のテスト,長い箇条書のテスト,長い箇条書のテスト,長い箇条書のテスト,長い箇条書のテスト,長い箇条書のテスト,長い箇条書のテスト,長い箇条書のテスト,長い箇条書のテスト,長い箇条書のテスト
  \item 長い箇条書のテスト,長い箇条書のテスト,長い箇条書のテスト,長い箇条書のテスト,長い箇条書のテスト,長い箇条書のテスト,長い箇条書のテスト,長い箇条書のテスト,長い箇条書のテスト,長い箇条書のテスト
\end{itemize}

\subsection{番号付き箇条書}

\begin{enumerate}
  \item 普通の箇条書1
  \item 普通の箇条書2
  \item 普通の箇条書3
  \item 長い箇条書のテスト,長い箇条書のテスト,長い箇条書のテスト,長い箇条書のテスト,長い箇条書のテスト,長い箇条書のテスト,長い箇条書のテスト,長い箇条書のテスト,長い箇条書のテスト,長い箇条書のテスト
  \item 長い箇条書のテスト,長い箇条書のテスト,長い箇条書のテスト,長い箇条書のテスト,長い箇条書のテスト,長い箇条書のテスト,長い箇条書のテスト,長い箇条書のテスト,長い箇条書のテスト,長い箇条書のテスト
\end{enumerate}

\begin{description}
  \item[hoge] 普通の箇条書1
  \item[foo] 普通の箇条書2
  \item[var] 普通の箇条書3
  \item[長い箇条書のタイトル] 長い箇条書のテスト,長い箇条書のテスト,長い箇条書のテスト,長い箇条書のテスト,長い箇条書のテスト,長い箇条書のテスト,長い箇条書のテスト,長い箇条書のテスト,長い箇条書のテスト,長い箇条書のテスト
  \item[長い箇条書のタイトル] 長い箇条書のテスト,長い箇条書のテスト,長い箇条書のテスト,長い箇条書のテスト,長い箇条書のテスト,長い箇条書のテスト,長い箇条書のテスト,長い箇条書のテスト,長い箇条書のテスト,長い箇条書のテスト
\end{description}

\subsubsection{subsubsectionのタイトル}

\paragraph{パラグラフ}

\subparagraph{サブパラグラフ}

文章の構造は,section, subsection, subsubsection, paragraph, subparagraph である.

%%
%% 本文 - ここまで
%%

%%%%%%%%%%%%%%%%%%%%%%%%%%%%%%%%%%%%%%%%%%%%%%%%%%%%%%%%%%%%%%%%%%%%%%%%

%%
%% 参考文献
%%

\begin{thebibliography}{9}	% {9}は文献に付ける通し番号の表示に必
				% 要な幅を指定している.10件以上になる
				% 場合には2桁の数({99}など)を指定する.
\bibitem{wusethesis}
  福安直樹,
  卒業論文スタイルファイル(和歌山大学システム工学部用),
  \url{https://github.com/fukuyasu/wuse_thesis}.

\bibitem{tex}
  Knuth, D.,
  Remarks to Celebrate the Publication of Computers \& Typesetting,
  TUGboat, Vol.7, No.2, pp.95--98, 1986.

\bibitem{latex}
  Lamport, L.,
  文書処理システム\LaTeXe{},
  ピアソン・エデュケーション,1999,
  \newblock{}阿瀬はる美 訳.

\bibitem{latex_j}
  奥村晴彦,\LaTeX{}入門 ---美文書作成のポイント---,技術評論社,1993.

\bibitem{latex2e}
  奥村晴彦,黒木裕介,[改定第6版] \LaTeXe~美文書作成入門,技術評論社,2013.

\bibitem{latexcomp}
  Goossens, M., Mittelbach, F. and Samarin, A.,
  The \LaTeX{}コンパニオン,アスキー出版局,1998,
  \newblock{}アスキー書籍編集部 監訳.

\bibitem{texwiki}
  \LaTeX 入門 --- \TeX{} Wiki,
  \url{https://texwiki.texjp.org/?LaTeX%E5%85%A5%E9%96%80},
  2021年12月3日閲覧.
\end{thebibliography}

%%%%%%%%%%%%%%%%%%%%%%%%%%%%%%%%%%%%%%%%%%%%%%%%%%%%%%%%%%%%%%%%%%%%%%%%

\end{document}
